\documentclass{aer}      % Specifies the document class

                             % The preamble begins here.
\usepackage[super,comma,round]{natbib}
\usepackage[figuresleft]{rotating}


\usepackage{txfonts}%

\jmonth{Month}
\jyear{2016}
\volume{000}
\issue{0000}
\doi{10.1017/aer.2016.1}
\historydate{Received DD MM YYYY; revised DD MM YYYY; accepted DD MM YYYY}

             % End of preamble and beginning of text.
\begin{document}



\title{Title title title title title title title title title title} %Declares the document's title.
\subtitle{Subtitle subtitle subtitle subtitle subtitle subtitle subtitle}


\markboth{Author et al.}{Short title Short title Short title Short title \ldots}



%Authors, affiliations address.

%Author with Email%
\author{A. Author One\email{abcd.xyz@mnop.edu}\\ B. Name Author Two\thanks{{\bf Author's notes}}}

\affil{Affiliation One\\ Department Name\\ City Name\\ Country Name}

\author{A. Author One, B. Author Two and C. Author Three}
\affil{Affiliation Two\\ Department Name\\ City Name\\ Country Name}

\author{A. Author One and C. Author Two}
\affil{Affiliation Three\\ Department Name\\ City Name\\ Country Name}

\maketitle                   % Produces the title.

\begin{abstract}
Dummy Text of abstract. Dummy Text of abstract. Dummy Text of abstract. Dummy Text of abstract. Dummy Text of abstract. Dummy Text of abstract. Dummy Text of abstract. Dummy Text of abstract. Dummy Text of abstract. Dummy Text of abstract. Dummy Text of abstract. Dummy Text of abstract. Dummy Text of abstract. Dummy Text of abstract. Dummy Text of abstract. Dummy Text of abstract. Dummy Text of abstract. Dummy Text of abstract. Dummy Text of abstract. Dummy Text of abstract. Dummy Text of abstract. Dummy Text of abstract. Dummy Text of abstract. Dummy Text of abstract. Dummy Text of abstract. Dummy Text of abstract. Dummy Text of abstract. Dummy Text of abstract.
\end{abstract}



\section*{NOMENCLATURE}
\begin{deflist}
\listterm{a}{definition text}
\listterm{a, b}{definition text definition text}
\listterm{\emph{APU}}{definition text definition text}
\listterm{ci}{definition text definition text}
\listterm{CT}{definition text}
\listterm{\emph{CAS}}{definition text definition text}
\listterm{E}{definition text}
\listterm{D}{definition text}
\listterm{f}{definition text}
\listterm{Fg}{definition text}
\listterm{F}{definition text}
\listterm{\emph{FCOM}}{definition text}
\end{deflist}


%% \begin{deflist}%[aaaa]
%% \listterm{T1}{BBBBBBB111}
%% \listterm{T2}{BBBBBBB222}
%% \listterm{T3\emph{T}3T4T5}{BBBBBBB333}
%% \end{deflist}

\subsection*{Greek Symbol}
\begin{deflist}
\listterm{$\alpha$}{definition text}
\listterm{$\beta$}{definition text definition text}
\listterm{$\gamma$}{definition text definition text}
\listterm{$\delta$}{definition text definition text}
\listterm{$\xi$}{definition text definition text}
\listterm{$\phi$}{definition text definition text definition text definition text definition text definition text definition text definition text definition text}
\listterm{$\psi$}{definition text definition text}
\end{deflist}


\section{FIRST-ORDER HEADING}% Head A
This is an example of dummy text. This is an example of dummy text. This is an example of dummy text. This is an example of dummy text. This is an example of dummy text. This is an example of dummy text. This is an example of dummy text. This is an example of dummy text. This is an example of dummy text. This is an example of dummy text. This is an example of dummy text. This is an example of dummy text. This is an example of dummy text. This is an example of dummy text. This is an example of dummy text. This is an example of dummy text. This is an example of dummy text. This is an example of dummy text. This is an example of dummy text. This is an example of dummy text. This is an example of dummy text. This is an example of dummy text. This is an example of dummy text. This is an example of dummy text.


\subsection{Second-Order Heading}% Head B
This is an example of dummy text. This is an example of dummy text. This is an example of dummy text.
This is an example of dummy text. This is an example of dummy text. This is an example of dummy text.
This is an example of dummy text. This is an example of dummy text. This is an example of dummy text.
This is an example of dummy text. This is an example of dummy text.\footnote{This is an example of first footnote.}


% Example of lists
\subsection{Lists} Here is an example of a numbered list:
\begin{enumerate}
\item List entry number 1 List entry number 1 List entry number 1 List entry number 1 List entry number 1 List entry number 1,
\item List entry number 2, List entry number 2, List entry number 2, List entry number 2, List entry number 2, List entry number 2,
\item List entry number 3,
\item List entry number 4, and
\item List entry number 5.
\end{enumerate}

The following is an example of an \emph{itemized}/\emph{bulleted}
list.

% itemize
\begin{itemize}
\item This is an example of bulleted listing.
\item This is an example of bulleted listing. This is an example of bulleted listing.
\item This is an example of bulleted listing.  This is an example of bulleted listing. This is an example of bulleted listing.
\item This is an example of bulleted listing. This is an example of bulleted listing. This is an example of bulleted listing. This is an example of bulleted listing.
\item This is an example of bulleted listing.
\end{itemize}

This is an example of dummy text. This is an example of dummy text. This is an example of dummy text. This is an example of dummy text. This is an example of dummy text. This is an example of dummy text. This is an example of dummy text. This is an example of dummy text. This is an example of dummy text. This is an example of dummy text. This is an example of dummy text. This is an example of dummy text. This is an example of dummy text. This is an example of dummy text. This is an example of dummy text. This is an example of dummy text. This is an example of dummy text. This is an example of dummy text. This is an example of dummy text. This is an example of dummy text. This is an example of dummy text. This is an example of dummy text. This is an example of dummy text. This is an example of dummy text.

\begin{itemize}
  \item The aforementioned methodology demonstrates significant applicability in the analytical framework.
\item The empirical evidence suggests a strong correlation between theoretical constructs and observed phenomena, as validated through multiple experimental iterations.
\item The proposed theoretical framework encompasses several interconnected components with hierarchical structure. Within this framework, a subsidiary enumeration delineates critical subcomponents:
         \begin{enumerate}
            \item This primary element represents the fundamental constituent of the enumerated taxonomy embedded within the hierarchical classification system.

            \item This secondary element further elucidates the taxonomic relationship.  
                  \LaTeX\ facilitates hierarchical nesting capabilities that exceed recommended structural complexity thresholds.
         \end{enumerate}
         The concluding segment of this hierarchical component maintains structural consistency with preceding elements, thereby preserving the integrity of the overarching taxonomic framework.
   \item The tertiary component completes the systematic classification schema.
\end{itemize}
\pagebreak

The default formatting of numbering can be changed as (1), (2), (3)\ldots, for example, 
\begin{arabiclist}
\item List entry one.
\item List entry two.
\item List entry three.
\item List entry four.
\item List entry five.
\end{arabiclist}


The default formatting of numbering can be changed as (i), (ii), (iii)\ldots, for example, 
\begin{romanlist}
\item List entry one.
\item List entry two.
\item List entry three.
\item List entry four.
\item List entry five.
\end{romanlist}


The default formatting of numbering can be changed as (a), (b), (c)\ldots, for example, 
\begin{alphalist}
\item List entry.
\item List entry two.
\item List entry three.
\item List entry four.
\item List entry five.
\end{alphalist}


The default formatting of numbering can be changed as I, II, III\ldots, for example, 
\begin{Romanlist}
\item List entry.
\item List entry two.
\item List entry three.
\item List entry four.
\item List entry five.
\end{Romanlist}


\subsection{Here is an example of extract}

\begin{extract}
This is an example text of quote or extract. This is an example text of quote or extract. This is an example text of quote or extract. This is an example text of quote or extract. This is an example text of quote or extract. This is an example text of quote or extract. This is an example text of quote or extract. This is an example text of quote or extract.
\end{extract}


This is dummy text. This is dummy text. This is dummy text. This is dummy text. This is dummy text. 

\subsection{Equations}
%Example of a single-line equation
Example of a single-line equation \ref{eqn1}.
\begin{equation}
a = b \ {\rm ((Single\ Equation\ Numbered))}\label{eqn1}
\end{equation}
%Example of multiple-line equation
Equations can also be multiple lines as shown in Equations \ref{eqn2} and \ref{eqn3}.
\begin{eqnarray}
c = 0 \ {\rm ((Multiple\  Lines, \ Numbered))}\label{eqn2}\\
ac = 0 \ {\rm ((Multiple \ Lines, \ Numbered))}\label{eqn3}
\end{eqnarray}
Unnumbered display equations single line:
   \[  {\Gamma}{\psi'} = x'' + y^{2} + z_{i}^{n}\]
Unnumbered display equations multiple lines:
\begin{eqnarray*}
c = 0 \ {\rm ((Multiple\  Lines, \ Unnumbered))}\nonumber\\
ac = 0 \ {\rm ((Multiple \ Lines, \ Unnumbered))}\nonumber
\end{eqnarray*}


\subsubsection{Third-Order Heading}%
This is dummy text. This is dummy text. This is dummy text. This is dummy text. This is dummy text.

This is dummy text. This is dummy text. This is dummy text. This is dummy text. This is dummy text. This is dummy text. This is dummy text. This is dummy text. This is dummy text. This is dummy text. This is dummy text. This is dummy text. This is dummy text. This is dummy text. This is dummy text. This is dummy text. This is dummy text. This is dummy text. This is dummy text. This is dummy text. This is dummy text. This is dummy text. This is dummy text. This is dummy text. This is dummy text. This is dummy text.

This is dummy text. This is dummy text. This is dummy text. This is dummy text. This is dummy text. This is dummy text. This is dummy text. This is dummy text. This is dummy text. This is dummy text. This is dummy text. This is dummy text. This is dummy text. This is dummy text. This is dummy text. This is dummy text. This is dummy text. This is dummy text. This is dummy text. This is dummy text. This is dummy text. This is dummy text. This is dummy text. This is dummy text. This is dummy text. This is dummy text. This is dummy text. This is dummy text.

\begin{figure}[b]
\begin{center}
\includegraphics[width=3in,height=1.3in]{fpo}%
\end{center}
\caption{This is an example of a figure caption. (\textit{a}) This is a description for part a,\break and (\textit{b}) this is a description for part b.}\label{fig1}
\end{figure}


This is dummy text. This is dummy text. This is dummy text. This is dummy text. This is dummy text. This is dummy text. This is dummy text. This is dummy text. This is dummy text. This is dummy text. This is dummy text. This is dummy text. This is dummy text. This is dummy text. This is dummy text. This is dummy text. This is dummy text. This is dummy text. This is dummy text. This is dummy text. This is dummy text. This is dummy text. This is dummy text. This is dummy text. This is dummy text. This is dummy text. This is dummy text. This is dummy text.

This is dummy text. This is dummy text. This is dummy text. This is dummy text. This is dummy text. This is dummy text. This is dummy text. This is dummy text. This is dummy text. This is dummy text. This is dummy text. This is dummy text. This is dummy text. This is dummy text. This is dummy text. This is dummy text. This is dummy text. This is dummy text. This is dummy text. This is dummy text. This is dummy text. This is dummy text. This is dummy text. This is dummy text. This is dummy text. This is dummy text. This is dummy text. This is dummy text.

\begin{figure}
\begin{center}
\includegraphics[width=2.5in,height=1.1in]{fpo}%
\end{center}
\caption{This is an example of a figure caption. (\textit{a}) This is a description for part a,\break and (\textit{b}) this is a description for part b.}\label{fig2}
\end{figure}

\begin{table}[tbp]
\caption{Table caption}
\label{tab1}
\centering
\begin{tabular}{lcccc}
    \textbf{Year}
  & \tch{1}{c}{b}{Single\\ outlet}  
  & \tch{1}{c}{b}{Small\\ multiple$^{a}$}  
  & \tch{1}{c}{b}{Large\\ multiple}  
  & \textbf{Total}   \\[6pt]
1982$^{b}$ & 98  & 129 & 620    & 847\\
1987 & 138 & 176 & 1000  & 1314\\
1991 & 173 & 248 & 1230  & 1651\\
1998 & 200 & 300 & 1500  & 2000\\
\end{tabular}
\begin{tabnote}
This is an example of unnumbered tablenote\\
$^{a}$This is an example of first numbered tablenote\\
$^{b}$This is an example of second numbered tablenote\\
\end{tabnote}
\end{table}


This is dummy text. This is dummy text. This is dummy text. This is dummy text. This is dummy text. This is dummy text. This is dummy text. This is dummy text. This is dummy text. This is dummy text. This is dummy text. This is dummy text. This is dummy text. This is dummy text. This is dummy text. This is dummy text. This is dummy text. This is dummy text. This is dummy text. This is dummy text. This is dummy text. This is dummy text. This is dummy text. This is dummy text. This is dummy text. This is dummy text. This is dummy text. This is dummy text.

Figure \ref{fig3} shows the dummy text. This is dummy text. This is dummy text. This is dummy text. This is dummy text. This is dummy text. This is dummy text. This is dummy text. This is dummy text. This is dummy text. This is dummy text. This is dummy text. This is dummy text. This is dummy text. This is dummy text. This is dummy text. This is dummy text. This is dummy text. This is dummy text. This is dummy text. This is dummy text. This is dummy text. This is dummy text. This is dummy text. This is dummy text. This is dummy text. This is dummy text. This is dummy text.

\begin{figure}
\begin{center}
\includegraphics[width=4in,height=2.2in]{fpo}%
\end{center}
\caption{This is an example of a figure caption. This is an example of a figure caption. This is an example of a figure caption. This is an example of a figure caption. This is an example of a figure caption. This is an example of a figure caption. (\textit{a}) This is a description for part a, and (\textit{b}) this is a description for part b.}\label{fig3}
\end{figure}


This is dummy text\footnote{This is an example of second footnote.}. This is dummy text. This is dummy text. This is dummy text. This is dummy text. This is dummy text. This is dummy text. This is dummy text. This is dummy text. This is dummy text. This is dummy text. This is dummy text. This is dummy text. This is dummy text. This is dummy text. This is dummy text. This is dummy text. This is dummy text. This is dummy text. This is dummy text. This is dummy text. This is dummy text. This is dummy text. This is dummy text. This is dummy text. This is dummy text. This is dummy text. This is dummy text. This is dummy text.

This is an example of theorem.
\begin{theorem}
This is an example of enunciation. This is an example of enunciation. This is an example of enunciation. This is an example of enunciation. This is an example of enunciation. This is an example of enunciation. 
       \[ {A}{B} = \sum_{i} a_{i} b_{i}\]
\end{theorem}



\section{FIRST-ORDER HEADING}% Head A%
\paragraph{Forth-Order Heading}%% Head D%


This is an example of Lemma
\begin{lemma}
This is an example of enunciation. This is an example of enunciation. This is an example of enunciation. This is an example of enunciation. This is an example of enunciation. This is an example of enunciation. 
       \[ {A}{B} = \sum_{i} a_{i} b_{i}\]
\end{lemma}

\begin{theorem}
This is an example of enunciation. This is an example of enunciation. This is an example of enunciation. This is an example of enunciation. This is an example of enunciation. This is an example of enunciation. 
       \[ {A}{B} = \sum_{i} a_{i} b_{i}\]
\end{theorem}


\begin{table}[tbp]
\tabcolsep20pt%used to increase column sep%
\caption{Table caption table caption table caption table caption table caption table\break caption table caption table caption table caption table caption table\break caption table caption table caption table caption}
\label{tab2}
\centering\begin{tabular}{@{}lcccc@{}}
&&\tch{2}{c}{b}{Span Col 3\&4} &\tch{2}{c}{b}{Span Col 5\& 6}\\
   \tch{1}{c}{b}{Column\\ One}
 & \tch{1}{c}{b}{Column\\ Two}
 & \tch{1}{c}{b}{Column\\ Three}
 & \tch{1}{c}{b}{Column\\ Four}
 & \tch{1}{c}{b}{Column\\ Five}
 & \tch{1}{c}{b}{Column\\ Six}\\[6pt]
6--31$+$G(d,p) &460 &2.12 &1.04 &8.63 &3.48\\
6-31$+$$+$G(d,p) &476 &2.24 &0.94 &9.22 &3.73\\
6-31$+$$+$G(df,p) &676 &3.71 &1.67 &16.36 &6.00\\
6-311$+$G(d,p) &572 &2.86 &1.29 &11.90 &4.56\\
6-311$+$$+$G(3df,2p) &1,060 &7.54 &3.25 &36.70 &11.45\\
\end{tabular}
\begin{tabnote}
$^{\rm a)}$Table Footnote\\ $^{\rm b)}$ Second Table Footnote.
\end{tabnote}
\end{table}


\begin{figure}[h]
\centering
\begin{minipage}[b]{.45\textwidth}
\centering
\includegraphics[width=1in,height=1.3in]{fpo}%
\caption{\hsize15pc This is an example of a figure caption. (\textit{a}) This is a description for part a,\break and (\textit{b}) this is a description for part b.}\label{fig5}
\end{minipage}\hfill
%
\begin{minipage}[b]{.45\textwidth}
\centering
 \includegraphics[width=2in,height=1.3in]{fpo}%
 \caption{This is an example of a figure caption. (\textit{a}) This is a description for part a,\break and (\textit{b}) this is a description for part b.}\label{fig6}
\end{minipage}
\end{figure}



\end{document}               % End of document.
